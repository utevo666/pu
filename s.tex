\begin{table}[h]
    \centering
    \begin{tabular}{|c|c|c|l|l|}
    \hline
    \textbf{Lp.} & \textbf{Państwo} & \textbf{Kontynent} & \textbf{Stolica} \\ \hline
    1   & Afganistan       & Azja      & Kabul     \\ \hline
    2   & Albania          & Europa    & Tirana    \\ \hline
    3   & Algieria         & Afryka    & Algier    \\ \hline
    4   & Andora           & Europa    & Andora    \\ \hline
    5   & Angola           & Afryka    & Luanda    \\ \hline
    6   & Antigua i Barbuda& Ameryka Północna & Saint John's \\ \hline
    7   & Arabia Saudyjska & Azja      & Rijad     \\ \hline
    8   & Argentyna        & Ameryka Południowa & Buenos Aires \\ \hline
    \end{tabular}
    \caption{}
    \label{tab:tabela1}
\end{table}
\section{Lista numerowana i punktowana}
To jest przepis na danie. Zwróć uwagę na punktowy i numerację.

\begin{itemize}
    \item {Składniki - danie:}
    \begin{enumerate}[label={\arabic*\textendash}, align=right]
        \item 500 g filetów białej ryby (np. dorsz, morszczuk, mintaj)
        \item 1,5 kg jabłek (najlepiej renet lub innych dobrych do pieczenia)
        \item 2 łyżki rodzynek
        \item 2 łyżki rumu
        \item 1/2 łyżeczki mielonego cynamonu
    \end{enumerate}
    \end{itemize}
\begin{itemize}
    \item {Składniki - sos:}
    \begin{enumerate}[label=\Alph*:]
        \item 90 g masła
        \item 125 g mąki
        \item 80 g cukru pudru (lub drobnego brązowego)
        \item Szczypta soli
    \end{enumerate}
    \end{itemize}
\begin{itemize}
    \item {Przygotowanie:}
    \begin{enumerate}
    \setcounter{enumi}{1}
    \stepcounter{enumi}
        \item Rodzynki włożyć do miseczki, zalać rumem, wymieszać i odstawić. Przygotować kruszonkę: na płaski talerz wyłożyć masło, mąkę, cukier i sól. Kroić wszystkie składniki nożem na jak najmniejsze kawałeczki, aż drobinki masła pokryją się mąką z cukrem.
            \stepcounter{enumi}
        \item Piekarnik nagrzać do 180 stopni C. Naczynie żaroodporne (ok. 20 x 30 cm) grubo wysmarować masłem. Jabłka obrać, pokroić na ćwiartki i wykroić gniazda nasienne. Każdą ćwiartkę jabłka pokroić na 4 cząstki i rozłożyć w naczyniu.
            \stepcounter{enumi}
        \item Rodzynki osączyć, ułożyć na jabłkach i oprószyć cynamonem. Owoce równomiernie posypać kruszonką. Wstawić do nagrzanego piekarnika i piec przez około 35 - 40 minut lub do czasu, aż kruszonka będzie dobrze zrumieniona i złocista.
    \end{enumerate}
\end{itemize}


























# 
\documentclass[10pt, a4paper, polish]{article}
\usepackage[utf8]{inputenc}
\usepackage[T1]{fontenc}
\usepackage[polish]{babel}
\usepackage{graphicx}
\usepackage{hyperref}
\usepackage{geometry}
\usepackage{array}
\usepackage{enumitem}


\geometry{a4paper, margin=2cm}

\title{Kolokwium LATEX - rząd P003}
\author{Imię i Nazwisko}
\date{30 listopada 2020}

\begin{document}

\maketitle

\section*{Wprowadzenie}
To jest wprowadzenie objaśniające nieco kolokwium. Zwróć uwagę, że jesteśmy w sekcji nienumerowanej. W sekcji tytułowej wpisz swoje prawdziwe imię i nazwisko. Jako datę wpisz ręcznie datę pisania kolokwium. W tytule znajduje się specjalny symbol. Ustaw czcionkę na 10 pkt, papier jako A4, dołącz pakiet polski. Spis treści jest na końcu. Wszystkie odsyłacze powinny być klikalne.

\tableofcontents

\section{Tabele}
W tej sekcji numerowanej testujemy tabele. Wykonaj poniższą tabelkę. Jest ona na dole strony. Kolumny poza pierwszą są po lewej. Pierwsza kolumna jest wyśrodkowana. Sama tabela też jest wyśrodkowana.

To jest odwołanie do tabeli \ref{tab:tabela1}.

\begin{table}[h]
    \centering
    \begin{tabular}{|c|c|c|l|l|}
    \hline
    \textbf{Lp.} & \textbf{Państwo} & \textbf{Kontynent} & \textbf{Stolica} \\ \hline
    1   & Afganistan       & Azja      & Kabul     \\ \hline
    2   & Albania          & Europa    & Tirana    \\ \hline
    3   & Algieria         & Afryka    & Algier    \\ \hline
    4   & Andora           & Europa    & Andora    \\ \hline
    5   & Angola           & Afryka    & Luanda    \\ \hline
    6   & Antigua i Barbuda& Ameryka Północna & Saint John's \\ \hline
    7   & Arabia Saudyjska & Azja      & Rijad     \\ \hline
    8   & Argentyna        & Ameryka Południowa & Buenos Aires \\ \hline
    \end{tabular}
    \caption{}
    \label{tab:tabela1}
\end{table}

\section{Zdjęcia}
Tutaj testujemy zdjęcia. Mamy parę tekstu lorem ipsum.
\begin{figure}[h]
    \centering
    \caption{Rysunek 1: Ten rysunek jest w bieżącym miejscu tekstu. Szerokość to 35 punktów.}
    \label{fig:rys1}
\end{figure}

\section{Lista numerowana i punktowana}
To jest przepis na danie. Zwróć uwagę na punktowy i numerację.

\begin{itemize}
    \item {Składniki - danie:}
    \begin{enumerate}[label={\arabic*\textendash}, align=right]
        \item 500 g filetów białej ryby (np. dorsz, morszczuk, mintaj)
        \item 1,5 kg jabłek (najlepiej renet lub innych dobrych do pieczenia)
        \item 2 łyżki rodzynek
        \item 2 łyżki rumu
        \item 1/2 łyżeczki mielonego cynamonu
    \end{enumerate}
    \end{itemize}
\begin{itemize}
    \item {Składniki - sos:}
    \begin{enumerate}[label=\Alph*:]
        \item 90 g masła
        \item 125 g mąki
        \item 80 g cukru pudru (lub drobnego brązowego)
        \item Szczypta soli
    \end{enumerate}
    \end{itemize}
\begin{itemize}
    \item {Przygotowanie:}
    \begin{enumerate}
    \setcounter{enumi}{1}
    \stepcounter{enumi}
        \item Rodzynki włożyć do miseczki, zalać rumem, wymieszać i odstawić. Przygotować kruszonkę: na płaski talerz wyłożyć masło, mąkę, cukier i sól. Kroić wszystkie składniki nożem na jak najmniejsze kawałeczki, aż drobinki masła pokryją się mąką z cukrem.
            \stepcounter{enumi}
        \item Piekarnik nagrzać do 180 stopni C. Naczynie żaroodporne (ok. 20 x 30 cm) grubo wysmarować masłem. Jabłka obrać, pokroić na ćwiartki i wykroić gniazda nasienne. Każdą ćwiartkę jabłka pokroić na 4 cząstki i rozłożyć w naczyniu.
            \stepcounter{enumi}
        \item Rodzynki osączyć, ułożyć na jabłkach i oprószyć cynamonem. Owoce równomiernie posypać kruszonką. Wstawić do nagrzanego piekarnika i piec przez około 35 - 40 minut lub do czasu, aż kruszonka będzie dobrze zrumieniona i złocista.
    \end{enumerate}
\end{itemize}

\section{Wzory matematyczne}
Tutaj testujemy wzory matematyczne. Wszystkie są wyśrodkowane w oddzielnym wierszu. Zwróć uwagę, że niektóre posiadają numery.

\[
\int \sin x \,dx = \sin^{-1} e^{\cos x} \cdot \frac{n-1}{n} + \frac{c}{n}
\]

\[
R^2 = \left( \frac{x}{y} \right) \left( \frac{z}{w} \right) = \frac{c^2}{F^2} \sin \theta \cos \theta \,dx
\]

Opracowano na podstawie \url{https://www.kwestiasmaku.com/zielony_srodek/salatki_jablka/jablka_kruszonka/przepis.html}.

\end{document}
